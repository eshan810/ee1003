\let\negmedspace\undefined
\let\negthickspace\undefined
\documentclass[journal]{IEEEtran}
\usepackage[a5paper, margin=10mm, onecolumn]{geometry}
%\usepackage{lmodern} % Ensure lmodern is loaded for pdflatex
\usepackage{tfrupee} % Include tfrupee package

\setlength{\headheight}{1cm} % Set the height of the header box
\setlength{\headsep}{0mm}  % Set the distance between the header box and the top of the text

\usepackage{gvv-book}
\usepackage{gvv}
\usepackage{cite}
\usepackage{amsmath,amssymb,amsfonts,amsthm}
\usepackage{algorithmic}
\usepackage{graphicx}
\usepackage{textcomp}
\usepackage{xcolor}
\usepackage{txfonts}
\usepackage{listings}
\usepackage{enumitem}
\usepackage{mathtools}
\usepackage{gensymb}
\usepackage{comment}
\usepackage[breaklinks=true]{hyperref}
\usepackage{tkz-euclide} 
\usepackage{listings}
% \usepackage{gvv}                                        
\def\inputGnumericTable{}                                 
\usepackage[latin1]{inputenc}                                
\usepackage{color}                                            
\usepackage{array}                                            
\usepackage{longtable}                                       
\usepackage{calc}                                             
\usepackage{multirow}                                         
\usepackage{hhline}                                           
\usepackage{ifthen}                                           
\usepackage{lscape}
\begin{document}

\bibliographystyle{IEEEtran}
\vspace{3cm}

\title{10.4.2.3}
\author{EE24BTECH11021 - Eshan Ray}

% \maketitle
% \newpage
% \bigskip
{\let\newpage\relax\maketitle}

\renewcommand{\thefigure}{\theenumi}
\renewcommand{\thetable}{\theenumi}
\setlength{\intextsep}{10pt} % Space between text and floats

\textbf{Question:}\\
Find two numbers whose sum is $27$ and product is $182$\\
\solution{
Let one of the numbers be $x$\\
So, the other number is $27-x$\\
Given,
\begin{align}
    x\brak{27-x} &= 182\\
    27x-x^2 &=182\\
    x^2 - 27x + 182 &=0\\
    \brak{x-13}\brak{x-14} &= 0\\
    \implies    x &= 13,14
\end{align}

So, the numbers are $13$ and $14$\\

\textbf{Computational Solution:}\\
Using Newton- Raphson Method we get,\\ \\
We start by taking an initial guess and then iteratively we us the following equation to find the roots of the quadratic equation :-
\begin{align}
    x_{n+1} &= x_n - \frac{f\brak{x_n}}{f\prime\brak{x_n}}\\
    f\brak{x} &= x^2 - 27x +182\\
    f\prime\brak{x} &= 2x - 27\\
    x_{n+1} &= x_n - \frac{x_{n}^2-27x_n+182}{2x_n-27}
\end{align}
After running the code, we obtained the following results:-
\begin{align}
    \text{Root-1:}\, 14.00000000\\
    \text{Root-2:}\, 13.00000000
\end{align}

\textbf{Alternate Method: Eigenvalues of Companion Matrix}\\
In this method, we find the roots of any polynomial of the form $x^n + a_{n-1}x^{n-1}\dots ax+a_0=0$ by finding the eigenvalues of the Companion Matrix $\brak{C}$ given below:-\\
\begin{align}
    C &= \myvec{0&1&0&\dots&0\\ 0&0&1&\dots&0\\ \vdots &\vdots &\vdots &\ddots&\vdots\\0&0&0&\vdots&1\\-a_0&-a_1&-a_2&\dots&-a_{n-1}}
\end{align}
For the Quadratic Equation $x^2-27x+182-0$, we get the following companion Matrix
\begin{align}
    C&=\myvec{0&1\\-182&27}
\end{align}
Using QR Decomposition with shifts to calculate the Eigenvalues of the companion Matrix we get the following eigenvalues/roots of the equation:-
\begin{align}
    \text{Roots of the quadratic equation:}\, [14.0, 13.0]
\end{align}
}
\end{document}



