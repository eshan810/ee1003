\documentclass{article}
\usepackage{graphicx}
\usepackage{amsmath}
\usepackage{hyperref}
\usepackage{listings}
\usepackage{xcolor}
\usepackage{longtable}

\title{Digital Clock}
\author{Eshan Ray\\ EE24BTECH11021}
\date{\today}

\begin{document}

\maketitle

\begin{abstract}
This document presents the design and implementation of an Arduino Uno-based digital clock featuring a countdown timer mode. The project utilizes an Arduino Uno and a 7-segment display for time representation. The system supports real-time clock functionality, user-configurable time adjustments, and a timer mode with auto-stop when time reaches zero.
\end{abstract}

\section{Introduction}
The purpose of this project is to develop a digital clock using an Arduino Uno. The clock includes features such as time display, time adjustment via push buttons, and a timer mode. The clock operates based on a millisecond timer interrupt and follows the 24-hour format.

\section{Hardware Components}
\begin{itemize}
\item Arduino Uno
\item 7-Segment Display (6 Digits)
\item Push Buttons (Hour, Minute, Second Increment/Decrement, Timer Mode, Reset)
\item Resistors and Capacitors
\item Power Supply (5V)
\end{itemize}

\section{Circuit Connections}
The following table outlines the connections between the Arduino Uno and various components:
\newpage
\begin{longtable}{|c|c|c|}
\hline
Component & Arduino Pin & Description \\
\hline
Hour Increment Button & PB0 & Increments the hour count \\
Hour Decrement Button & PB1 & Decrements the hour count \\
Minute Increment Button & PC5 & Increments the minute count \\
Minute Decrement Button & PC3 & Decrements the minute count \\
Second Increment Button & PD0 & Increments the second count \\
Timer Mode Button & PC4 & Toggles timer mode on/off \\
Reset Button & PC2 & Resets or restores time settings \\
7-Segment Display & PD2, PD3, PD4, PD5 & Used for digit selection (decoder) \\
Display Enable Lines & PB5, PB4, PC1, PC0, PB3, PD7 & Selects active digit \\
\hline
\end{longtable}

Each button is connected to a specific pin on the Arduino Uno with internal pull-up resistors enabled. The display is multiplexed using a decoder circuit and selection lines to cycle through the six digits.

\section{Software Implementation}
The project is implemented in C using the AVR-GCC compiler. The system structure includes:
\begin{itemize}
\item Millisecond timer interrupt for real-time updates.
\item Button handling with debounce and long-press detection.
\item Time management functions for incrementing, decrementing, and resetting time.
\item Display multiplexing for 7-segment display using a decoder.
\end{itemize}

\section{Code Implementation}
The main functionalities are defined in the following sections:

\subsection{Timer Interrupt for Millis Counter}
The system maintains a millisecond counter using the Timer0 Compare Match interrupt.

\begin{lstlisting}[language=C, caption=Millis Timer Interrupt, basicstyle=\ttfamily, keywordstyle=\color{blue}]
ISR(TIMER0\_COMPA\_vect) {
millis\_count++;
}

unsigned long currentMillis(void) {
unsigned long m;
cli();
m = millis\_count;
sei();
return m;
}
\end{lstlisting}

\subsection{Button Handling}
Each button is mapped to specific microcontroller pins. A debounce mechanism prevents false triggers.

\begin{lstlisting}[language=C, caption=Button Checking Function]
void checkButtons(void) {
static unsigned long holdStart[NUM\_BUTTONS] = {0};
void (*actions[])() = {incrementHour, decrementHour, incrementMinute,
decrementMinute, toggleTimerMode, incrementSecond, resetClock};
// Process button states and handle actions
}
\end{lstlisting}

\subsection{Time Update Functions}
The system supports both time increment and decrement operations, ensuring proper rollover at 24-hour and 60-minute limits.

\begin{lstlisting}[language=C, caption=Increment Time]
void incrementTime(void) {
seconds++;
if(seconds > 59) {
seconds = 0;
minutes++;
if(minutes > 59) {
minutes = 0;
hours++;
if(hours > 23) {
hours = 0;
}
}
}
}
\end{lstlisting}

\subsection{Display Multiplexing}
The 7-segment display is controlled using a decoder and port-based switching. Since the Arduino Uno has a limited number of I/O pins, the display multiplexing is done by enabling only one display at a time while sending the corresponding digit data.

\begin{lstlisting}[language=C, caption=Show Digit Function]
void showDigit(uint8\_t digit, uint8\_t displayIndex) {
*decoderPort &= ~((1 << PD2) | (1 << PD5) | (1 << PD4) | (1 << PD3));
if(digit & 0x01) *decoderPort |= (1 << PD2);
if(digit & 0x02) *decoderPort |= (1 << PD5);
if(digit & 0x04) *decoderPort |= (1 << PD4);
if(digit & 0x08) *decoderPort |= (1 << PD3);
}
\end{lstlisting}

\section{Results and Testing}
The clock was successfully tested for accurate timekeeping and responsiveness to user inputs. The timer mode correctly halts at zero, and button presses reliably adjust time settings.

\section{Conclusion}
This project demonstrates an effective implementation of a digital clock with a countdown timer using an Arduino Uno. Future improvements could include  an OLED display for enhanced readability.

\section{Project Repository}
The source code and additional details can be found at: \\ 
\href{https://github.com/eshan810/ee1003/blob/main/Hardware/Clock/codes/clock.c}{https://github.com/eshan810/ee1003/blob/main/Hardware/Clock/codes/clock.c}


\end{document}
